\documentclass[11pt]{article}
\usepackage{amsmath, amsfonts, amsthm}
\usepackage[utf8]{inputenc}
\usepackage[T2A]{fontenc}
\usepackage[russian]{babel}
\usepackage{fontspec}
\setmainfont{Times New Roman}
\begin{document}

\section*{Вариант 1}
Даны точки A(-12, 3, -0), B(21, 8, 3), C(-20, -6, -29), M(-5, -5, -7), K(-8, -21, -14), P(-3, 2, 20)

Составить уравнение плоскостей $ABC$ и $MKP$, найти линию пересечения плоскостей или установить их параллельность.\\
Составить систему параметрических уравнений прямых $MK$ и $CN$, прямая $CN$ перпендикулярна плоскости $ABC$. Найти точку пересечения $MK$ и $CN$.

\section*{Вариант 2}
Даны точки A(7, 18, 5), B(6, 27, -5), C(9, -3, -4), M(12, 11, 2), K(-9, -8, 4), P(29, 15, -20)

Составить уравнение плоскостей $ABC$ и $MKP$, найти линию пересечения плоскостей или установить их параллельность.\\
Составить систему параметрических уравнений прямых $MK$ и $CN$, прямая $CN$ перпендикулярна плоскости $ABC$. Найти точку пересечения $MK$ и $CN$.

\section*{Вариант 3}
Даны точки A(21, -7, -2), B(19, -26, 1), C(8, 6, -27), M(12, 1, -24), K(-10, 27, -5), P(9, -15, -5)

Составить уравнение плоскостей $ABC$ и $MKP$, найти линию пересечения плоскостей или установить их параллельность.\\
Составить систему параметрических уравнений прямых $MK$ и $CN$, прямая $CN$ перпендикулярна плоскости $ABC$. Найти точку пересечения $MK$ и $CN$.

\section*{Вариант 4}
Даны точки A(-0, 27, 23), B(0, 6, -11), C(-8, 13, 29), M(-3, 4, 21), K(-15, 11, -29), P(3, 1, 6)

Составить уравнение плоскостей $ABC$ и $MKP$, найти линию пересечения плоскостей или установить их параллельность.\\
Составить систему параметрических уравнений прямых $MK$ и $CN$, прямая $CN$ перпендикулярна плоскости $ABC$. Найти точку пересечения $MK$ и $CN$.

\section*{Вариант 5}
Даны точки A(-6, -21, -13), B(6, -10, -0), C(-2, -17, -5), M(10, -2, 23), K(0, 28, -2), P(-21, -28, -15)

Составить уравнение плоскостей $ABC$ и $MKP$, найти линию пересечения плоскостей или установить их параллельность.\\
Составить систему параметрических уравнений прямых $MK$ и $CN$, прямая $CN$ перпендикулярна плоскости $ABC$. Найти точку пересечения $MK$ и $CN$.

\section*{Вариант 6}
Даны точки A(25, 5, -5), B(13, 25, -9), C(-11, -4, 20), M(-3, 26, -21), K(-3, 7, -26), P(-7, -9, 8)

Составить уравнение плоскостей $ABC$ и $MKP$, найти линию пересечения плоскостей или установить их параллельность.\\
Составить систему параметрических уравнений прямых $MK$ и $CN$, прямая $CN$ перпендикулярна плоскости $ABC$. Найти точку пересечения $MK$ и $CN$.

\section*{Вариант 7}
Даны точки A(-25, -5, 8), B(-11, -17, 4), C(-20, 22, 1), M(-9, -10, -5), K(8, 11, 28), P(23, 9, 6)

Составить уравнение плоскостей $ABC$ и $MKP$, найти линию пересечения плоскостей или установить их параллельность.\\
Составить систему параметрических уравнений прямых $MK$ и $CN$, прямая $CN$ перпендикулярна плоскости $ABC$. Найти точку пересечения $MK$ и $CN$.

\section*{Вариант 8}
Даны точки A(-17, -22, -28), B(9, -21, -1), C(3, 3, -1), M(-2, 2, 13), K(6, -0, 7), P(-24, -8, 8)

Составить уравнение плоскостей $ABC$ и $MKP$, найти линию пересечения плоскостей или установить их параллельность.\\
Составить систему параметрических уравнений прямых $MK$ и $CN$, прямая $CN$ перпендикулярна плоскости $ABC$. Найти точку пересечения $MK$ и $CN$.

\section*{Вариант 9}
Даны точки A(-13, 6, -9), B(17, 6, 24), C(19, 1, -4), M(-21, -11, -8), K(23, 17, 13), P(27, 16, -7)

Составить уравнение плоскостей $ABC$ и $MKP$, найти линию пересечения плоскостей или установить их параллельность.\\
Составить систему параметрических уравнений прямых $MK$ и $CN$, прямая $CN$ перпендикулярна плоскости $ABC$. Найти точку пересечения $MK$ и $CN$.

\section*{Вариант 10}
Даны точки A(0, 24, 27), B(19, -5, -21), C(10, 2, 9), M(18, -23, 6), K(24, -14, 17), P(25, -7, 7)

Составить уравнение плоскостей $ABC$ и $MKP$, найти линию пересечения плоскостей или установить их параллельность.\\
Составить систему параметрических уравнений прямых $MK$ и $CN$, прямая $CN$ перпендикулярна плоскости $ABC$. Найти точку пересечения $MK$ и $CN$.

\section*{Вариант 11}
Даны точки A(-6, -5, 20), B(-29, -7, 1), C(-2, -29, -3), M(-17, 7, -9), K(11, 17, -0), P(11, 1, 18)

Составить уравнение плоскостей $ABC$ и $MKP$, найти линию пересечения плоскостей или установить их параллельность.\\
Составить систему параметрических уравнений прямых $MK$ и $CN$, прямая $CN$ перпендикулярна плоскости $ABC$. Найти точку пересечения $MK$ и $CN$.

\section*{Вариант 12}
Даны точки A(-14, -17, -8), B(15, -14, 14), C(29, 12, 0), M(-16, -4, -3), K(4, -24, -13), P(6, -9, 22)

Составить уравнение плоскостей $ABC$ и $MKP$, найти линию пересечения плоскостей или установить их параллельность.\\
Составить систему параметрических уравнений прямых $MK$ и $CN$, прямая $CN$ перпендикулярна плоскости $ABC$. Найти точку пересечения $MK$ и $CN$.

\section*{Вариант 13}
Даны точки A(-24, -3, -0), B(-1, -9, 2), C(-2, -7, -11), M(4, -0, -1), K(21, 1, 11), P(8, -12, 25)

Составить уравнение плоскостей $ABC$ и $MKP$, найти линию пересечения плоскостей или установить их параллельность.\\
Составить систему параметрических уравнений прямых $MK$ и $CN$, прямая $CN$ перпендикулярна плоскости $ABC$. Найти точку пересечения $MK$ и $CN$.

\section*{Вариант 14}
Даны точки A(-24, 5, 27), B(13, 8, -15), C(9, 13, -1), M(-1, -21, -27), K(8, 6, -15), P(-3, -2, -14)

Составить уравнение плоскостей $ABC$ и $MKP$, найти линию пересечения плоскостей или установить их параллельность.\\
Составить систему параметрических уравнений прямых $MK$ и $CN$, прямая $CN$ перпендикулярна плоскости $ABC$. Найти точку пересечения $MK$ и $CN$.

\section*{Вариант 15}
Даны точки A(7, -28, -18), B(20, 9, 14), C(7, -7, -1), M(-0, 9, -8), K(16, 7, -20), P(16, -2, -4)

Составить уравнение плоскостей $ABC$ и $MKP$, найти линию пересечения плоскостей или установить их параллельность.\\
Составить систему параметрических уравнений прямых $MK$ и $CN$, прямая $CN$ перпендикулярна плоскости $ABC$. Найти точку пересечения $MK$ и $CN$.

\section*{Вариант 16}
Даны точки A(23, -3, 8), B(25, -22, -4), C(26, 18, 28), M(-26, -13, -16), K(-28, 10, -5), P(-11, 5, 8)

Составить уравнение плоскостей $ABC$ и $MKP$, найти линию пересечения плоскостей или установить их параллельность.\\
Составить систему параметрических уравнений прямых $MK$ и $CN$, прямая $CN$ перпендикулярна плоскости $ABC$. Найти точку пересечения $MK$ и $CN$.

\section*{Вариант 17}
Даны точки A(15, 4, 28), B(24, 26, -15), C(7, 2, -0), M(-6, -26, 28), K(23, -7, 1), P(-16, -18, 7)

Составить уравнение плоскостей $ABC$ и $MKP$, найти линию пересечения плоскостей или установить их параллельность.\\
Составить систему параметрических уравнений прямых $MK$ и $CN$, прямая $CN$ перпендикулярна плоскости $ABC$. Найти точку пересечения $MK$ и $CN$.

\section*{Вариант 18}
Даны точки A(-20, -1, 8), B(-26, 23, -29), C(-9, -3, -7), M(21, 23, -21), K(-8, 18, 7), P(0, 18, 21)

Составить уравнение плоскостей $ABC$ и $MKP$, найти линию пересечения плоскостей или установить их параллельность.\\
Составить систему параметрических уравнений прямых $MK$ и $CN$, прямая $CN$ перпендикулярна плоскости $ABC$. Найти точку пересечения $MK$ и $CN$.

\section*{Вариант 19}
Даны точки A(-10, -0, -5), B(20, -2, 9), C(-6, -14, -22), M(-9, 3, -8), K(8, 1, 9), P(-2, -2, 22)

Составить уравнение плоскостей $ABC$ и $MKP$, найти линию пересечения плоскостей или установить их параллельность.\\
Составить систему параметрических уравнений прямых $MK$ и $CN$, прямая $CN$ перпендикулярна плоскости $ABC$. Найти точку пересечения $MK$ и $CN$.

\section*{Вариант 20}
Даны точки A(-28, -2, -7), B(5, 25, 10), C(-28, -8, 28), M(-21, 7, -20), K(13, -9, 15), P(27, -18, -19)

Составить уравнение плоскостей $ABC$ и $MKP$, найти линию пересечения плоскостей или установить их параллельность.\\
Составить систему параметрических уравнений прямых $MK$ и $CN$, прямая $CN$ перпендикулярна плоскости $ABC$. Найти точку пересечения $MK$ и $CN$.

\section*{Вариант 21}
Даны точки A(20, -24, -8), B(-27, -20, -25), C(16, -9, 21), M(-6, -15, 6), K(-18, 3, 24), P(-11, -11, -6)

Составить уравнение плоскостей $ABC$ и $MKP$, найти линию пересечения плоскостей или установить их параллельность.\\
Составить систему параметрических уравнений прямых $MK$ и $CN$, прямая $CN$ перпендикулярна плоскости $ABC$. Найти точку пересечения $MK$ и $CN$.

\section*{Вариант 22}
Даны точки A(29, 5, -1), B(1, -23, -6), C(8, -0, 16), M(5, -6, -28), K(-1, 3, 0), P(-2, 21, 1)

Составить уравнение плоскостей $ABC$ и $MKP$, найти линию пересечения плоскостей или установить их параллельность.\\
Составить систему параметрических уравнений прямых $MK$ и $CN$, прямая $CN$ перпендикулярна плоскости $ABC$. Найти точку пересечения $MK$ и $CN$.

\section*{Вариант 23}
Даны точки A(-5, -5, -17), B(27, 28, -18), C(16, 29, 29), M(24, -9, 4), K(8, -7, -1), P(3, 14, -21)

Составить уравнение плоскостей $ABC$ и $MKP$, найти линию пересечения плоскостей или установить их параллельность.\\
Составить систему параметрических уравнений прямых $MK$ и $CN$, прямая $CN$ перпендикулярна плоскости $ABC$. Найти точку пересечения $MK$ и $CN$.

\section*{Вариант 24}
Даны точки A(-18, 0, 25), B(6, -1, 20), C(12, -0, 9), M(16, -10, -0), K(-28, -7, 29), P(9, -2, -22)

Составить уравнение плоскостей $ABC$ и $MKP$, найти линию пересечения плоскостей или установить их параллельность.\\
Составить систему параметрических уравнений прямых $MK$ и $CN$, прямая $CN$ перпендикулярна плоскости $ABC$. Найти точку пересечения $MK$ и $CN$.

\section*{Вариант 25}
Даны точки A(-22, 2, -22), B(-20, 4, 20), C(8, -11, 27), M(-27, -9, 2), K(-22, 24, -17), P(-28, 7, 14)

Составить уравнение плоскостей $ABC$ и $MKP$, найти линию пересечения плоскостей или установить их параллельность.\\
Составить систему параметрических уравнений прямых $MK$ и $CN$, прямая $CN$ перпендикулярна плоскости $ABC$. Найти точку пересечения $MK$ и $CN$.

\section*{Вариант 26}
Даны точки A(-1, -13, 8), B(-14, -17, 9), C(-8, -4, 4), M(-20, -4, 0), K(6, 11, 9), P(3, 10, -13)

Составить уравнение плоскостей $ABC$ и $MKP$, найти линию пересечения плоскостей или установить их параллельность.\\
Составить систему параметрических уравнений прямых $MK$ и $CN$, прямая $CN$ перпендикулярна плоскости $ABC$. Найти точку пересечения $MK$ и $CN$.

\section*{Вариант 27}
Даны точки A(-29, -20, -2), B(0, 18, -8), C(10, 4, 20), M(16, -5, -17), K(-17, -9, -9), P(-26, -5, -14)

Составить уравнение плоскостей $ABC$ и $MKP$, найти линию пересечения плоскостей или установить их параллельность.\\
Составить систему параметрических уравнений прямых $MK$ и $CN$, прямая $CN$ перпендикулярна плоскости $ABC$. Найти точку пересечения $MK$ и $CN$.

\section*{Вариант 28}
Даны точки A(-22, 14, -21), B(-4, -6, 6), C(-13, 2, 9), M(27, 8, 0), K(28, -8, -1), P(-11, 0, -22)

Составить уравнение плоскостей $ABC$ и $MKP$, найти линию пересечения плоскостей или установить их параллельность.\\
Составить систему параметрических уравнений прямых $MK$ и $CN$, прямая $CN$ перпендикулярна плоскости $ABC$. Найти точку пересечения $MK$ и $CN$.

\section*{Вариант 29}
Даны точки A(17, -2, 26), B(-19, 16, -1), C(-5, -2, 6), M(-8, 4, 14), K(-13, 9, -4), P(-0, -19, 6)

Составить уравнение плоскостей $ABC$ и $MKP$, найти линию пересечения плоскостей или установить их параллельность.\\
Составить систему параметрических уравнений прямых $MK$ и $CN$, прямая $CN$ перпендикулярна плоскости $ABC$. Найти точку пересечения $MK$ и $CN$.

\section*{Вариант 30}
Даны точки A(-3, 26, -9), B(-27, 10, -17), C(-16, 12, 0), M(23, -8, -25), K(-3, -13, -6), P(-4, 14, 8)

Составить уравнение плоскостей $ABC$ и $MKP$, найти линию пересечения плоскостей или установить их параллельность.\\
Составить систему параметрических уравнений прямых $MK$ и $CN$, прямая $CN$ перпендикулярна плоскости $ABC$. Найти точку пересечения $MK$ и $CN$.

\section*{Вариант 31}
Даны точки A(-5, -22, -10), B(-16, 26, 3), C(-13, 2, 5), M(-1, 25, 21), K(-22, 6, -12), P(-0, 22, -12)

Составить уравнение плоскостей $ABC$ и $MKP$, найти линию пересечения плоскостей или установить их параллельность.\\
Составить систему параметрических уравнений прямых $MK$ и $CN$, прямая $CN$ перпендикулярна плоскости $ABC$. Найти точку пересечения $MK$ и $CN$.

\section*{Вариант 32}
Даны точки A(1, 3, -0), B(21, 16, 9), C(17, -9, -5), M(15, -20, 1), K(-20, 2, -21), P(24, -2, -4)

Составить уравнение плоскостей $ABC$ и $MKP$, найти линию пересечения плоскостей или установить их параллельность.\\
Составить систему параметрических уравнений прямых $MK$ и $CN$, прямая $CN$ перпендикулярна плоскости $ABC$. Найти точку пересечения $MK$ и $CN$.

\section*{Вариант 33}
Даны точки A(13, -16, 10), B(9, -2, -2), C(-18, 2, -9), M(-22, 15, -6), K(1, 1, 8), P(-7, 28, -22)

Составить уравнение плоскостей $ABC$ и $MKP$, найти линию пересечения плоскостей или установить их параллельность.\\
Составить систему параметрических уравнений прямых $MK$ и $CN$, прямая $CN$ перпендикулярна плоскости $ABC$. Найти точку пересечения $MK$ и $CN$.

\section*{Вариант 34}
Даны точки A(5, 17, -16), B(22, -16, -19), C(-12, 9, 0), M(7, -1, -4), K(-29, 11, 2), P(4, 21, -9)

Составить уравнение плоскостей $ABC$ и $MKP$, найти линию пересечения плоскостей или установить их параллельность.\\
Составить систему параметрических уравнений прямых $MK$ и $CN$, прямая $CN$ перпендикулярна плоскости $ABC$. Найти точку пересечения $MK$ и $CN$.

\section*{Вариант 35}
Даны точки A(1, -0, -3), B(10, 2, -29), C(-6, 2, 7), M(-24, -10, 11), K(-8, -7, -3), P(11, 21, 9)

Составить уравнение плоскостей $ABC$ и $MKP$, найти линию пересечения плоскостей или установить их параллельность.\\
Составить систему параметрических уравнений прямых $MK$ и $CN$, прямая $CN$ перпендикулярна плоскости $ABC$. Найти точку пересечения $MK$ и $CN$.

\section*{Вариант 36}
Даны точки A(-24, -24, -8), B(-28, -4, -1), C(0, -25, -14), M(-18, -2, 6), K(-17, -17, 24), P(-6, -6, 17)

Составить уравнение плоскостей $ABC$ и $MKP$, найти линию пересечения плоскостей или установить их параллельность.\\
Составить систему параметрических уравнений прямых $MK$ и $CN$, прямая $CN$ перпендикулярна плоскости $ABC$. Найти точку пересечения $MK$ и $CN$.

\section*{Вариант 37}
Даны точки A(4, 5, 13), B(8, -23, 5), C(-10, -8, -13), M(-12, 8, -1), K(4, 4, -29), P(-29, 3, 15)

Составить уравнение плоскостей $ABC$ и $MKP$, найти линию пересечения плоскостей или установить их параллельность.\\
Составить систему параметрических уравнений прямых $MK$ и $CN$, прямая $CN$ перпендикулярна плоскости $ABC$. Найти точку пересечения $MK$ и $CN$.

\section*{Вариант 38}
Даны точки A(12, 16, -18), B(18, -27, -2), C(21, 2, -29), M(-1, 0, 6), K(4, -3, -21), P(-4, 25, -28)

Составить уравнение плоскостей $ABC$ и $MKP$, найти линию пересечения плоскостей или установить их параллельность.\\
Составить систему параметрических уравнений прямых $MK$ и $CN$, прямая $CN$ перпендикулярна плоскости $ABC$. Найти точку пересечения $MK$ и $CN$.

\section*{Вариант 39}
Даны точки A(6, 1, -8), B(5, -19, 14), C(3, -0, 1), M(23, -1, 21), K(-17, 19, 0), P(-14, 3, 29)

Составить уравнение плоскостей $ABC$ и $MKP$, найти линию пересечения плоскостей или установить их параллельность.\\
Составить систему параметрических уравнений прямых $MK$ и $CN$, прямая $CN$ перпендикулярна плоскости $ABC$. Найти точку пересечения $MK$ и $CN$.

\section*{Вариант 40}
Даны точки A(6, 18, 0), B(25, 11, -7), C(-16, -4, 0), M(7, 2, 8), K(28, 18, 5), P(-7, -29, 17)

Составить уравнение плоскостей $ABC$ и $MKP$, найти линию пересечения плоскостей или установить их параллельность.\\
Составить систему параметрических уравнений прямых $MK$ и $CN$, прямая $CN$ перпендикулярна плоскости $ABC$. Найти точку пересечения $MK$ и $CN$.

\section*{Вариант 41}
Даны точки A(-6, -1, -4), B(15, -18, 0), C(11, 7, -15), M(-6, 1, -26), K(6, -11, 12), P(-0, -2, -0)

Составить уравнение плоскостей $ABC$ и $MKP$, найти линию пересечения плоскостей или установить их параллельность.\\
Составить систему параметрических уравнений прямых $MK$ и $CN$, прямая $CN$ перпендикулярна плоскости $ABC$. Найти точку пересечения $MK$ и $CN$.

\section*{Вариант 42}
Даны точки A(12, -1, 2), B(8, 23, -6), C(16, -1, 18), M(18, -21, -21), K(15, 25, 4), P(-2, 25, 24)

Составить уравнение плоскостей $ABC$ и $MKP$, найти линию пересечения плоскостей или установить их параллельность.\\
Составить систему параметрических уравнений прямых $MK$ и $CN$, прямая $CN$ перпендикулярна плоскости $ABC$. Найти точку пересечения $MK$ и $CN$.

\section*{Вариант 43}
Даны точки A(-1, -24, -12), B(-8, 1, 17), C(6, -10, 23), M(-15, -3, -1), K(-6, 18, -20), P(-16, 16, -15)

Составить уравнение плоскостей $ABC$ и $MKP$, найти линию пересечения плоскостей или установить их параллельность.\\
Составить систему параметрических уравнений прямых $MK$ и $CN$, прямая $CN$ перпендикулярна плоскости $ABC$. Найти точку пересечения $MK$ и $CN$.

\section*{Вариант 44}
Даны точки A(-9, -14, 7), B(11, 16, -8), C(-1, -6, -15), M(-4, 11, 11), K(-2, -13, 20), P(-10, 27, -23)

Составить уравнение плоскостей $ABC$ и $MKP$, найти линию пересечения плоскостей или установить их параллельность.\\
Составить систему параметрических уравнений прямых $MK$ и $CN$, прямая $CN$ перпендикулярна плоскости $ABC$. Найти точку пересечения $MK$ и $CN$.

\section*{Вариант 45}
Даны точки A(-27, -25, 12), B(-0, 5, 26), C(13, -7, 6), M(8, -17, -9), K(-11, -9, -3), P(-15, -15, 17)

Составить уравнение плоскостей $ABC$ и $MKP$, найти линию пересечения плоскостей или установить их параллельность.\\
Составить систему параметрических уравнений прямых $MK$ и $CN$, прямая $CN$ перпендикулярна плоскости $ABC$. Найти точку пересечения $MK$ и $CN$.

\section*{Вариант 46}
Даны точки A(0, 29, -9), B(-16, -8, 12), C(28, 0, 5), M(-24, 0, -6), K(0, 9, -7), P(-7, 16, 27)

Составить уравнение плоскостей $ABC$ и $MKP$, найти линию пересечения плоскостей или установить их параллельность.\\
Составить систему параметрических уравнений прямых $MK$ и $CN$, прямая $CN$ перпендикулярна плоскости $ABC$. Найти точку пересечения $MK$ и $CN$.

\section*{Вариант 47}
Даны точки A(-7, 4, 15), B(-7, 0, -8), C(-13, 27, 20), M(-17, -10, -6), K(25, -3, 7), P(22, 25, 9)

Составить уравнение плоскостей $ABC$ и $MKP$, найти линию пересечения плоскостей или установить их параллельность.\\
Составить систему параметрических уравнений прямых $MK$ и $CN$, прямая $CN$ перпендикулярна плоскости $ABC$. Найти точку пересечения $MK$ и $CN$.

\section*{Вариант 48}
Даны точки A(22, 22, -20), B(1, 22, -7), C(-9, -1, 19), M(22, -24, 19), K(26, 6, 9), P(-16, 4, 9)

Составить уравнение плоскостей $ABC$ и $MKP$, найти линию пересечения плоскостей или установить их параллельность.\\
Составить систему параметрических уравнений прямых $MK$ и $CN$, прямая $CN$ перпендикулярна плоскости $ABC$. Найти точку пересечения $MK$ и $CN$.

\section*{Вариант 49}
Даны точки A(18, 13, -4), B(6, 3, 3), C(-12, 6, -12), M(2, 0, -1), K(-27, 12, 12), P(0, 17, 28)

Составить уравнение плоскостей $ABC$ и $MKP$, найти линию пересечения плоскостей или установить их параллельность.\\
Составить систему параметрических уравнений прямых $MK$ и $CN$, прямая $CN$ перпендикулярна плоскости $ABC$. Найти точку пересечения $MK$ и $CN$.

\section*{Вариант 50}
Даны точки A(-21, -21, 2), B(-2, -3, -1), C(-19, 20, 10), M(28, 0, -29), K(-4, -29, -9), P(-25, 7, -13)

Составить уравнение плоскостей $ABC$ и $MKP$, найти линию пересечения плоскостей или установить их параллельность.\\
Составить систему параметрических уравнений прямых $MK$ и $CN$, прямая $CN$ перпендикулярна плоскости $ABC$. Найти точку пересечения $MK$ и $CN$.

\section*{Вариант 51}
Даны точки A(-5, 22, 1), B(9, 9, 1), C(-25, -19, 15), M(10, -28, 24), K(-27, -6, -3), P(-3, 6, -19)

Составить уравнение плоскостей $ABC$ и $MKP$, найти линию пересечения плоскостей или установить их параллельность.\\
Составить систему параметрических уравнений прямых $MK$ и $CN$, прямая $CN$ перпендикулярна плоскости $ABC$. Найти точку пересечения $MK$ и $CN$.

\section*{Вариант 52}
Даны точки A(6, 0, 1), B(-20, 1, 8), C(7, -9, -12), M(26, 2, -9), K(8, -5, -15), P(21, -5, -14)

Составить уравнение плоскостей $ABC$ и $MKP$, найти линию пересечения плоскостей или установить их параллельность.\\
Составить систему параметрических уравнений прямых $MK$ и $CN$, прямая $CN$ перпендикулярна плоскости $ABC$. Найти точку пересечения $MK$ и $CN$.

\section*{Вариант 53}
Даны точки A(9, -8, -10), B(-1, -22, -18), C(1, -6, -0), M(16, -6, -12), K(-7, -23, 20), P(-27, -8, -14)

Составить уравнение плоскостей $ABC$ и $MKP$, найти линию пересечения плоскостей или установить их параллельность.\\
Составить систему параметрических уравнений прямых $MK$ и $CN$, прямая $CN$ перпендикулярна плоскости $ABC$. Найти точку пересечения $MK$ и $CN$.

\section*{Вариант 54}
Даны точки A(21, -3, -6), B(18, 1, 2), C(-6, -15, -1), M(5, 0, 5), K(-10, -2, 27), P(0, -7, 1)

Составить уравнение плоскостей $ABC$ и $MKP$, найти линию пересечения плоскостей или установить их параллельность.\\
Составить систему параметрических уравнений прямых $MK$ и $CN$, прямая $CN$ перпендикулярна плоскости $ABC$. Найти точку пересечения $MK$ и $CN$.

\section*{Вариант 55}
Даны точки A(0, -22, 29), B(-8, -28, 10), C(5, -14, 9), M(17, -22, 11), K(7, 19, 9), P(-22, -11, -2)

Составить уравнение плоскостей $ABC$ и $MKP$, найти линию пересечения плоскостей или установить их параллельность.\\
Составить систему параметрических уравнений прямых $MK$ и $CN$, прямая $CN$ перпендикулярна плоскости $ABC$. Найти точку пересечения $MK$ и $CN$.

\section*{Вариант 56}
Даны точки A(18, -0, 7), B(4, 28, -10), C(-11, -28, -4), M(-19, 7, -6), K(6, -4, -2), P(8, 1, 15)

Составить уравнение плоскостей $ABC$ и $MKP$, найти линию пересечения плоскостей или установить их параллельность.\\
Составить систему параметрических уравнений прямых $MK$ и $CN$, прямая $CN$ перпендикулярна плоскости $ABC$. Найти точку пересечения $MK$ и $CN$.

\section*{Вариант 57}
Даны точки A(0, 6, 25), B(-14, 6, -7), C(8, 0, 23), M(-2, -17, -26), K(-29, -3, 9), P(14, 2, 28)

Составить уравнение плоскостей $ABC$ и $MKP$, найти линию пересечения плоскостей или установить их параллельность.\\
Составить систему параметрических уравнений прямых $MK$ и $CN$, прямая $CN$ перпендикулярна плоскости $ABC$. Найти точку пересечения $MK$ и $CN$.

\section*{Вариант 58}
Даны точки A(-7, 8, -21), B(-6, -9, -14), C(-10, -2, 2), M(4, 7, -0), K(-20, -2, -18), P(-13, -16, -25)

Составить уравнение плоскостей $ABC$ и $MKP$, найти линию пересечения плоскостей или установить их параллельность.\\
Составить систему параметрических уравнений прямых $MK$ и $CN$, прямая $CN$ перпендикулярна плоскости $ABC$. Найти точку пересечения $MK$ и $CN$.

\section*{Вариант 59}
Даны точки A(2, -6, -27), B(-26, -24, -8), C(29, -27, 1), M(7, 2, 11), K(-4, -13, 7), P(27, -25, -3)

Составить уравнение плоскостей $ABC$ и $MKP$, найти линию пересечения плоскостей или установить их параллельность.\\
Составить систему параметрических уравнений прямых $MK$ и $CN$, прямая $CN$ перпендикулярна плоскости $ABC$. Найти точку пересечения $MK$ и $CN$.

\section*{Вариант 60}
Даны точки A(-28, 8, 12), B(-1, -15, -9), C(-29, 9, -8), M(28, 2, -26), K(-29, -2, 7), P(-28, -10, 9)

Составить уравнение плоскостей $ABC$ и $MKP$, найти линию пересечения плоскостей или установить их параллельность.\\
Составить систему параметрических уравнений прямых $MK$ и $CN$, прямая $CN$ перпендикулярна плоскости $ABC$. Найти точку пересечения $MK$ и $CN$.

\section*{Вариант 61}
Даны точки A(6, -8, 10), B(-0, 9, 7), C(17, 23, 15), M(-28, 4, 12), K(-2, 2, 24), P(3, -9, 9)

Составить уравнение плоскостей $ABC$ и $MKP$, найти линию пересечения плоскостей или установить их параллельность.\\
Составить систему параметрических уравнений прямых $MK$ и $CN$, прямая $CN$ перпендикулярна плоскости $ABC$. Найти точку пересечения $MK$ и $CN$.

\section*{Вариант 62}
Даны точки A(8, -3, 7), B(0, 2, -8), C(-0, -0, 5), M(-2, -13, 22), K(-11, 3, 16), P(9, -17, -8)

Составить уравнение плоскостей $ABC$ и $MKP$, найти линию пересечения плоскостей или установить их параллельность.\\
Составить систему параметрических уравнений прямых $MK$ и $CN$, прямая $CN$ перпендикулярна плоскости $ABC$. Найти точку пересечения $MK$ и $CN$.

\section*{Вариант 63}
Даны точки A(-14, 2, 21), B(0, 12, 24), C(0, 4, -15), M(8, 23, 0), K(29, -11, -4), P(-1, 23, -1)

Составить уравнение плоскостей $ABC$ и $MKP$, найти линию пересечения плоскостей или установить их параллельность.\\
Составить систему параметрических уравнений прямых $MK$ и $CN$, прямая $CN$ перпендикулярна плоскости $ABC$. Найти точку пересечения $MK$ и $CN$.

\section*{Вариант 64}
Даны точки A(-29, -5, 7), B(-19, -9, -18), C(-21, -20, 0), M(2, 25, 7), K(3, 9, -28), P(-3, -15, -5)

Составить уравнение плоскостей $ABC$ и $MKP$, найти линию пересечения плоскостей или установить их параллельность.\\
Составить систему параметрических уравнений прямых $MK$ и $CN$, прямая $CN$ перпендикулярна плоскости $ABC$. Найти точку пересечения $MK$ и $CN$.

\section*{Вариант 65}
Даны точки A(0, 24, -9), B(21, -15, -0), C(-7, 2, -1), M(-27, -14, -4), K(-11, -1, -2), P(-12, 24, -8)

Составить уравнение плоскостей $ABC$ и $MKP$, найти линию пересечения плоскостей или установить их параллельность.\\
Составить систему параметрических уравнений прямых $MK$ и $CN$, прямая $CN$ перпендикулярна плоскости $ABC$. Найти точку пересечения $MK$ и $CN$.

\section*{Вариант 66}
Даны точки A(-27, -22, 21), B(1, 9, -14), C(9, -6, -2), M(7, -2, -7), K(7, -2, -0), P(5, 28, -20)

Составить уравнение плоскостей $ABC$ и $MKP$, найти линию пересечения плоскостей или установить их параллельность.\\
Составить систему параметрических уравнений прямых $MK$ и $CN$, прямая $CN$ перпендикулярна плоскости $ABC$. Найти точку пересечения $MK$ и $CN$.

\section*{Вариант 67}
Даны точки A(7, -13, 2), B(13, -6, 23), C(4, 27, 12), M(-1, -5, 20), K(-13, 27, -2), P(-23, -10, -6)

Составить уравнение плоскостей $ABC$ и $MKP$, найти линию пересечения плоскостей или установить их параллельность.\\
Составить систему параметрических уравнений прямых $MK$ и $CN$, прямая $CN$ перпендикулярна плоскости $ABC$. Найти точку пересечения $MK$ и $CN$.

\section*{Вариант 68}
Даны точки A(6, 25, -4), B(18, -7, -6), C(-19, 8, 9), M(9, 25, 20), K(-18, -12, 15), P(-5, 6, 6)

Составить уравнение плоскостей $ABC$ и $MKP$, найти линию пересечения плоскостей или установить их параллельность.\\
Составить систему параметрических уравнений прямых $MK$ и $CN$, прямая $CN$ перпендикулярна плоскости $ABC$. Найти точку пересечения $MK$ и $CN$.

\section*{Вариант 69}
Даны точки A(-0, 29, 17), B(10, 14, -1), C(4, 18, -2), M(13, 5, 17), K(-17, 5, 22), P(-18, -6, -18)

Составить уравнение плоскостей $ABC$ и $MKP$, найти линию пересечения плоскостей или установить их параллельность.\\
Составить систему параметрических уравнений прямых $MK$ и $CN$, прямая $CN$ перпендикулярна плоскости $ABC$. Найти точку пересечения $MK$ и $CN$.

\section*{Вариант 70}
Даны точки A(-8, 1, -1), B(-4, 9, -16), C(14, 11, 6), M(-3, -7, 17), K(28, -0, 7), P(13, -17, 6)

Составить уравнение плоскостей $ABC$ и $MKP$, найти линию пересечения плоскостей или установить их параллельность.\\
Составить систему параметрических уравнений прямых $MK$ и $CN$, прямая $CN$ перпендикулярна плоскости $ABC$. Найти точку пересечения $MK$ и $CN$.

\section*{Вариант 71}
Даны точки A(-28, -16, -5), B(12, 10, -28), C(-4, 22, 14), M(2, -16, -25), K(-12, -28, 1), P(19, 4, -17)

Составить уравнение плоскостей $ABC$ и $MKP$, найти линию пересечения плоскостей или установить их параллельность.\\
Составить систему параметрических уравнений прямых $MK$ и $CN$, прямая $CN$ перпендикулярна плоскости $ABC$. Найти точку пересечения $MK$ и $CN$.

\section*{Вариант 72}
Даны точки A(-27, -6, 3), B(-8, 5, 24), C(-12, -6, 20), M(13, -5, 13), K(15, 8, -1), P(-1, 7, -27)

Составить уравнение плоскостей $ABC$ и $MKP$, найти линию пересечения плоскостей или установить их параллельность.\\
Составить систему параметрических уравнений прямых $MK$ и $CN$, прямая $CN$ перпендикулярна плоскости $ABC$. Найти точку пересечения $MK$ и $CN$.

\section*{Вариант 73}
Даны точки A(7, -5, 20), B(27, -3, 10), C(1, -8, -7), M(5, 14, -7), K(0, -5, -10), P(29, 4, 5)

Составить уравнение плоскостей $ABC$ и $MKP$, найти линию пересечения плоскостей или установить их параллельность.\\
Составить систему параметрических уравнений прямых $MK$ и $CN$, прямая $CN$ перпендикулярна плоскости $ABC$. Найти точку пересечения $MK$ и $CN$.

\section*{Вариант 74}
Даны точки A(-8, -7, -4), B(22, -4, 1), C(-7, 0, -8), M(-6, -0, -15), K(-5, -6, -6), P(-9, -5, -29)

Составить уравнение плоскостей $ABC$ и $MKP$, найти линию пересечения плоскостей или установить их параллельность.\\
Составить систему параметрических уравнений прямых $MK$ и $CN$, прямая $CN$ перпендикулярна плоскости $ABC$. Найти точку пересечения $MK$ и $CN$.

\section*{Вариант 75}
Даны точки A(-4, -0, -20), B(1, -8, -15), C(19, 23, -2), M(-19, -6, 17), K(4, 25, 9), P(-1, 7, 1)

Составить уравнение плоскостей $ABC$ и $MKP$, найти линию пересечения плоскостей или установить их параллельность.\\
Составить систему параметрических уравнений прямых $MK$ и $CN$, прямая $CN$ перпендикулярна плоскости $ABC$. Найти точку пересечения $MK$ и $CN$.

\section*{Вариант 76}
Даны точки A(-28, 0, -22), B(-28, 0, -12), C(-23, 6, 22), M(-9, 17, -6), K(4, -6, -16), P(7, -12, 0)

Составить уравнение плоскостей $ABC$ и $MKP$, найти линию пересечения плоскостей или установить их параллельность.\\
Составить систему параметрических уравнений прямых $MK$ и $CN$, прямая $CN$ перпендикулярна плоскости $ABC$. Найти точку пересечения $MK$ и $CN$.

\section*{Вариант 77}
Даны точки A(-7, 4, 4), B(-24, -8, -2), C(-27, -5, -12), M(9, 7, 4), K(3, 7, 1), P(-26, 1, 28)

Составить уравнение плоскостей $ABC$ и $MKP$, найти линию пересечения плоскостей или установить их параллельность.\\
Составить систему параметрических уравнений прямых $MK$ и $CN$, прямая $CN$ перпендикулярна плоскости $ABC$. Найти точку пересечения $MK$ и $CN$.

\section*{Вариант 78}
Даны точки A(-27, -3, -2), B(-20, 24, -7), C(22, 7, -13), M(6, -12, -10), K(2, -25, -22), P(-27, 21, -8)

Составить уравнение плоскостей $ABC$ и $MKP$, найти линию пересечения плоскостей или установить их параллельность.\\
Составить систему параметрических уравнений прямых $MK$ и $CN$, прямая $CN$ перпендикулярна плоскости $ABC$. Найти точку пересечения $MK$ и $CN$.

\section*{Вариант 79}
Даны точки A(-19, -9, 0), B(29, -11, 19), C(17, -22, -17), M(-15, 3, 11), K(-23, 6, -8), P(-14, 4, -5)

Составить уравнение плоскостей $ABC$ и $MKP$, найти линию пересечения плоскостей или установить их параллельность.\\
Составить систему параметрических уравнений прямых $MK$ и $CN$, прямая $CN$ перпендикулярна плоскости $ABC$. Найти точку пересечения $MK$ и $CN$.

\section*{Вариант 80}
Даны точки A(5, 20, -22), B(-1, -7, -5), C(-29, 1, 5), M(-0, 9, -3), K(-8, -16, 23), P(-12, 6, 3)

Составить уравнение плоскостей $ABC$ и $MKP$, найти линию пересечения плоскостей или установить их параллельность.\\
Составить систему параметрических уравнений прямых $MK$ и $CN$, прямая $CN$ перпендикулярна плоскости $ABC$. Найти точку пересечения $MK$ и $CN$.

\section*{Вариант 81}
Даны точки A(0, 29, 19), B(6, 22, 5), C(-19, -28, 2), M(-0, -15, -2), K(-2, -9, 9), P(-22, -16, 10)

Составить уравнение плоскостей $ABC$ и $MKP$, найти линию пересечения плоскостей или установить их параллельность.\\
Составить систему параметрических уравнений прямых $MK$ и $CN$, прямая $CN$ перпендикулярна плоскости $ABC$. Найти точку пересечения $MK$ и $CN$.

\section*{Вариант 82}
Даны точки A(-2, -29, -12), B(-9, 6, -0), C(-8, -3, 9), M(25, -1, -7), K(25, 26, 6), P(-1, 18, 9)

Составить уравнение плоскостей $ABC$ и $MKP$, найти линию пересечения плоскостей или установить их параллельность.\\
Составить систему параметрических уравнений прямых $MK$ и $CN$, прямая $CN$ перпендикулярна плоскости $ABC$. Найти точку пересечения $MK$ и $CN$.

\section*{Вариант 83}
Даны точки A(-17, -6, 15), B(9, 10, -5), C(2, -7, 6), M(3, 24, -0), K(-18, -28, 16), P(1, 3, 7)

Составить уравнение плоскостей $ABC$ и $MKP$, найти линию пересечения плоскостей или установить их параллельность.\\
Составить систему параметрических уравнений прямых $MK$ и $CN$, прямая $CN$ перпендикулярна плоскости $ABC$. Найти точку пересечения $MK$ и $CN$.

\section*{Вариант 84}
Даны точки A(-10, 19, 13), B(-1, -6, -18), C(-12, -8, 6), M(-11, 14, 7), K(-3, 27, -8), P(11, 8, 0)

Составить уравнение плоскостей $ABC$ и $MKP$, найти линию пересечения плоскостей или установить их параллельность.\\
Составить систему параметрических уравнений прямых $MK$ и $CN$, прямая $CN$ перпендикулярна плоскости $ABC$. Найти точку пересечения $MK$ и $CN$.

\section*{Вариант 85}
Даны точки A(-26, 1, 0), B(5, -14, 25), C(12, 24, 25), M(6, -7, -14), K(-22, -0, 21), P(20, -16, 27)

Составить уравнение плоскостей $ABC$ и $MKP$, найти линию пересечения плоскостей или установить их параллельность.\\
Составить систему параметрических уравнений прямых $MK$ и $CN$, прямая $CN$ перпендикулярна плоскости $ABC$. Найти точку пересечения $MK$ и $CN$.

\section*{Вариант 86}
Даны точки A(1, -7, 23), B(-8, 3, 22), C(-1, -15, -7), M(18, 17, 5), K(-3, -23, 5), P(27, 22, 9)

Составить уравнение плоскостей $ABC$ и $MKP$, найти линию пересечения плоскостей или установить их параллельность.\\
Составить систему параметрических уравнений прямых $MK$ и $CN$, прямая $CN$ перпендикулярна плоскости $ABC$. Найти точку пересечения $MK$ и $CN$.

\section*{Вариант 87}
Даны точки A(-5, -19, 15), B(-4, -6, 1), C(3, -25, -12), M(-7, -25, -1), K(23, -8, -5), P(9, -3, 19)

Составить уравнение плоскостей $ABC$ и $MKP$, найти линию пересечения плоскостей или установить их параллельность.\\
Составить систему параметрических уравнений прямых $MK$ и $CN$, прямая $CN$ перпендикулярна плоскости $ABC$. Найти точку пересечения $MK$ и $CN$.

\section*{Вариант 88}
Даны точки A(6, 9, -6), B(18, -7, 0), C(-4, -10, -1), M(21, -2, -3), K(-2, 4, 14), P(-21, 9, 7)

Составить уравнение плоскостей $ABC$ и $MKP$, найти линию пересечения плоскостей или установить их параллельность.\\
Составить систему параметрических уравнений прямых $MK$ и $CN$, прямая $CN$ перпендикулярна плоскости $ABC$. Найти точку пересечения $MK$ и $CN$.

\section*{Вариант 89}
Даны точки A(-2, -28, -5), B(23, -1, 11), C(9, 0, 2), M(17, 0, 24), K(-5, 9, -1), P(-4, 24, -1)

Составить уравнение плоскостей $ABC$ и $MKP$, найти линию пересечения плоскостей или установить их параллельность.\\
Составить систему параметрических уравнений прямых $MK$ и $CN$, прямая $CN$ перпендикулярна плоскости $ABC$. Найти точку пересечения $MK$ и $CN$.

\section*{Вариант 90}
Даны точки A(1, -26, -4), B(13, 6, 26), C(-10, -3, -8), M(1, 15, 5), K(-8, 24, 6), P(-0, 26, -22)

Составить уравнение плоскостей $ABC$ и $MKP$, найти линию пересечения плоскостей или установить их параллельность.\\
Составить систему параметрических уравнений прямых $MK$ и $CN$, прямая $CN$ перпендикулярна плоскости $ABC$. Найти точку пересечения $MK$ и $CN$.

\section*{Вариант 91}
Даны точки A(0, -21, -7), B(-4, -22, -5), C(9, -1, -17), M(20, -16, 1), K(-23, -11, -4), P(4, 20, -10)

Составить уравнение плоскостей $ABC$ и $MKP$, найти линию пересечения плоскостей или установить их параллельность.\\
Составить систему параметрических уравнений прямых $MK$ и $CN$, прямая $CN$ перпендикулярна плоскости $ABC$. Найти точку пересечения $MK$ и $CN$.

\section*{Вариант 92}
Даны точки A(24, -20, 5), B(24, 15, -22), C(3, -7, 7), M(9, -2, -24), K(6, 7, 11), P(-4, 9, -13)

Составить уравнение плоскостей $ABC$ и $MKP$, найти линию пересечения плоскостей или установить их параллельность.\\
Составить систему параметрических уравнений прямых $MK$ и $CN$, прямая $CN$ перпендикулярна плоскости $ABC$. Найти точку пересечения $MK$ и $CN$.

\section*{Вариант 93}
Даны точки A(8, -29, 1), B(-15, -5, 16), C(-9, 20, -22), M(-7, 3, 3), K(2, 1, 13), P(4, -28, -19)

Составить уравнение плоскостей $ABC$ и $MKP$, найти линию пересечения плоскостей или установить их параллельность.\\
Составить систему параметрических уравнений прямых $MK$ и $CN$, прямая $CN$ перпендикулярна плоскости $ABC$. Найти точку пересечения $MK$ и $CN$.

\section*{Вариант 94}
Даны точки A(0, -7, -8), B(-11, 6, -23), C(-12, -7, 11), M(-15, 17, -5), K(-8, 4, -15), P(-3, -3, -29)

Составить уравнение плоскостей $ABC$ и $MKP$, найти линию пересечения плоскостей или установить их параллельность.\\
Составить систему параметрических уравнений прямых $MK$ и $CN$, прямая $CN$ перпендикулярна плоскости $ABC$. Найти точку пересечения $MK$ и $CN$.

\section*{Вариант 95}
Даны точки A(27, 6, 17), B(24, 25, 14), C(-4, -11, 21), M(-6, 15, 3), K(-1, -1, -6), P(25, -6, -28)

Составить уравнение плоскостей $ABC$ и $MKP$, найти линию пересечения плоскостей или установить их параллельность.\\
Составить систему параметрических уравнений прямых $MK$ и $CN$, прямая $CN$ перпендикулярна плоскости $ABC$. Найти точку пересечения $MK$ и $CN$.

\section*{Вариант 96}
Даны точки A(1, 11, -0), B(1, 19, -24), C(-7, 8, 0), M(4, 27, 9), K(9, -8, -25), P(8, 10, -6)

Составить уравнение плоскостей $ABC$ и $MKP$, найти линию пересечения плоскостей или установить их параллельность.\\
Составить систему параметрических уравнений прямых $MK$ и $CN$, прямая $CN$ перпендикулярна плоскости $ABC$. Найти точку пересечения $MK$ и $CN$.

\section*{Вариант 97}
Даны точки A(6, 5, -29), B(12, -8, -28), C(10, 10, 28), M(19, 3, 1), K(1, -11, -25), P(-28, -12, -17)

Составить уравнение плоскостей $ABC$ и $MKP$, найти линию пересечения плоскостей или установить их параллельность.\\
Составить систему параметрических уравнений прямых $MK$ и $CN$, прямая $CN$ перпендикулярна плоскости $ABC$. Найти точку пересечения $MK$ и $CN$.

\section*{Вариант 98}
Даны точки A(-0, -12, -0), B(17, -5, -1), C(-24, -19, -29), M(-9, 23, 19), K(4, 7, -11), P(7, 4, -5)

Составить уравнение плоскостей $ABC$ и $MKP$, найти линию пересечения плоскостей или установить их параллельность.\\
Составить систему параметрических уравнений прямых $MK$ и $CN$, прямая $CN$ перпендикулярна плоскости $ABC$. Найти точку пересечения $MK$ и $CN$.

\section*{Вариант 99}
Даны точки A(9, 9, -0), B(-18, 12, 16), C(25, -21, 8), M(-24, 20, -17), K(-5, -19, -7), P(0, -24, -9)

Составить уравнение плоскостей $ABC$ и $MKP$, найти линию пересечения плоскостей или установить их параллельность.\\
Составить систему параметрических уравнений прямых $MK$ и $CN$, прямая $CN$ перпендикулярна плоскости $ABC$. Найти точку пересечения $MK$ и $CN$.

\section*{Вариант 100}
Даны точки A(16, -6, 11), B(-16, -18, -4), C(-0, 14, 16), M(1, 2, -8), K(3, -7, 16), P(6, 2, 29)

Составить уравнение плоскостей $ABC$ и $MKP$, найти линию пересечения плоскостей или установить их параллельность.\\
Составить систему параметрических уравнений прямых $MK$ и $CN$, прямая $CN$ перпендикулярна плоскости $ABC$. Найти точку пересечения $MK$ и $CN$.
\end{document}
% generated with task generator